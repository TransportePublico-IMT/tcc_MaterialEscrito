\begin{resumo}[Abstract]
	\begin{otherlanguage*}{english}
	\indent
	\par This project aims to create a dashboard that monitors the demand of vehicles in bus lines of SPTrans, the company responsible for buses in the city of São Paulo. For this monitoring, public data provided by SPTrans was used, such as bus routes, GPS data, lines and stops, for example, which will be enriched with other information, like dates and locations of events, status of train and subway lines, weather information and traffic around the city.
	\par Although a lot of public transport data is public available, there isn’t a system to aggregate all this information that would be able to help people who use this kind of transport on a daily basis with a overview about the lines condition. The present work intends to contribute positively to improve the user experience and bring a more transparent and complete view of the service.
	\par During the development, tools such as Python, Django, Django-REST, Celery and Amazon Web Services (AWS) were used, which made it possible to create a dashboard and all its features, from the acquisition of external data, to the manipulation of that data and deployment of the platform in a cloud platform.
	\\
	\\
	\textbf{Key-words}: ~\ifthenelse{\equal{\KeyWordB}{}}{}{\KeyWordB.~}\ifthenelse{\equal{\KeyWordC}{}}{}{\KeyWordC.~}\ifthenelse{\equal{\KeyWordD}{}}{}{\KeyWordD.~}\ifthenelse{\equal{\KeyWordE}{}}{}{\KeyWordE.~}\ifthenelse{\equal{\KeyWordF}{}}{}{\KeyWordF.~}
	\end{otherlanguage*}
\end{resumo}
