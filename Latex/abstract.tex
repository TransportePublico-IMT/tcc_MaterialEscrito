\begin{resumo}[Abstract]
	\begin{otherlanguage*}{english}
	\indent
	\par The project in question aims to create a dashboard that monitors the demands of vehicle fleets, in this case the bus lines of SPTrans, the company responsible for buses in the city of São Paulo. For this monitoring, public data provided by SPTrans will be used, such as bus routes, GPS data, lines and stops, for example, which will be enriched with other information, such as dates and locations of events, situation of train and subway lines, weather information and traffic in the city.
\par With this information in mind, the project also aims to develop an artificial intelligence model that can be used by the SPTrans operational team to help control demand from fleets, indicating whether more or less buses are needed on each line. The data presented by the model can also be used by ordinary citizens who seek greater transparency to observe the situation of the lines and potential factors that may alter the normal flow of public transport.
\par Considering that currently many public transportation data are public, but there is no centralized platform for this information that is able to help people with an overview of the conditions of the bus lines, the present work intends to contribute positively to improve the user experience and bring a more transparent view of the service as a whole.
\par To build the system, tools such as Python, Django, Django-REST, Celery and Amazon Web Services (AWS) were used, which enabled the development of the dashboard and all its features, from the acquisition of external data, to the manipulation of that data and deployment of the platform in the cloud.
		\textbf{Key-words}: \KeyWordA.~\ifthenelse{\equal{\KeyWordB}{}}{}{\KeyWordB.~}\ifthenelse{\equal{\KeyWordC}{}}{}{\KeyWordC.~}\ifthenelse{\equal{\KeyWordD}{}}{}{\KeyWordD.~}\ifthenelse{\equal{\KeyWordE}{}}{}{\KeyWordE.~}\ifthenelse{\equal{\KeyWordF}{}}{}{\KeyWordF.~}
	\end{otherlanguage*}
\end{resumo}
