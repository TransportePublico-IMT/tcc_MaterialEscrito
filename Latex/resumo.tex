\begin{resumo}
\indent
\par O projeto em questão tem como objetivo a criação de um dashboard que monitora as demandas das   frotas   de veículos, no caso as linhas de ônibus da SPTrans, empresa responsável pelos ônibus da cidade de São Paulo. Para esse monitoramento, serão utilizados dados públicos fornecidos pela SPTrans, como rotas dos ônibus, dados de GPS, linhas e paradas, por exemplo, que serão enriquecidos com outras informações, como datas e locais de eventos, situação das linhas de trens e metrôs, informações meteorológicas e o trânsito na cidade.
\par Com posse dessas informações, o projeto também visa o desenvolvimento de um modelo de inteligência artificial que poderá ser utilizado pela equipe operacional da SPTrans para auxiliar o controle de demanda das frotas, indicando se são necessários mais ou menos ônibus em cada linha. Os dados apresentados pelo modelo também poderão ser utilizados por cidadãos comuns que procuram uma maior transparência para observar a situação das linhas e potenciais fatores que poderão alterar o fluxo normal do transporte público.
\par Considerando que atualmente muitos dados do transporte coletivo são públicos, mas não existe uma plataforma centralizadora dessas informações que seja capaz de auxiliar as pessoas no que diz respeito a uma visão geral das condições das linhas de ônibus, o presente trabalho pretende contribuir positivamente para melhorar a experiência dos usuários e trazer uma visão mais transparente do serviço como um todo.
\par Para a construção do sistema foram utilizadas ferramentas como Python, Django, Django-REST, Celery e Amazon Web Services (AWS), que possibilitaram o desenvolvimento do dashboard e todas suas funcionalidades, desde a aquisição de dados externos, até a manipulação desses dados e implantação da plataforma na nuvem.
\\
\\
\textbf{Palavras-chaves}: \PalavraChaveA.~\ifthenelse{\equal{\PalavraChaveB}{}}{}{\PalavraChaveB.~}\ifthenelse{\equal{\PalavraChaveC}{}}{}{\PalavraChaveC.~}\ifthenelse{\equal{\PalavraChaveD}{}}{}{\PalavraChaveD.~}\ifthenelse{\equal{\PalavraChaveE}{}}{}{\PalavraChaveE.~}\ifthenelse{\equal{\PalavraChaveF}{}}{}{\PalavraChaveF.~}

\end{resumo}
