\begin{resumo}
\indent
\par O projeto tem como objetivo a criação de um painel de visualização (\textit{dashboard}) que monitora as demandas das   frotas   de veículos, no caso as linhas de ônibus da SPTrans, empresa responsável pelos ônibus da cidade de São Paulo. Para esse monitoramento, foram utilizados dados públicos fornecidos pela SPTrans, como rotas dos ônibus, dados de GPS, linhas e paradas, por exemplo, que foram enriquecidos com outras informações, como datas e locais de eventos, situação das linhas de trens e metrôs, informações meteorológicas e o trânsito na cidade. Considerando que atualmente muitos dados do transporte coletivo são públicos, mas não existe uma plataforma centralizadora dessas informações que seja capaz de auxiliar as pessoas no que diz respeito a uma visão geral das condições das linhas de ônibus, o presente trabalho pretende contribuir positivamente para melhorar a experiência dos usuários e trazer uma visão mais transparente do serviço como um todo. Para a construção do sistema foram utilizadas ferramentas como Python, Django, Django-REST, Celery e Amazon Web Services (AWS), que possibilitaram o desenvolvimento do \textit{dashboard} e todas suas funcionalidades, desde a aquisição de dados externos, até a manipulação desses dados e implantação da plataforma na nuvem.
\\
\\
\textbf{Palavras-chaves}: \PalavraChaveA.~\ifthenelse{\equal{\PalavraChaveB}{}}{}{\PalavraChaveB.~}\ifthenelse{\equal{\PalavraChaveC}{}}{}{\PalavraChaveC.~}\ifthenelse{\equal{\PalavraChaveD}{}}{}{\PalavraChaveD.~}\ifthenelse{\equal{\PalavraChaveE}{}}{}{\PalavraChaveE.~}\ifthenelse{\equal{\PalavraChaveF}{}}{}{\PalavraChaveF.~}

\end{resumo}
