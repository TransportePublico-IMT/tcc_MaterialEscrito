\chapter{Revisão bibliográfica}
\label{Cap:RevisaoBibliografica}
\newcommand{\WidthAlgumaCoisa}{6.5 cm}

% Capítulo 2: Revisão Bibliográfica
Este capítulo detalha toda revisão literária aplicada ao projeto,  abordando assim conceitos teóricos, tecnologias e contexto histórico.

\section{Dashboard}

\par O Dashboard, também chamado de painel de controle, é uma ferramenta que auxilia os gestores a terem uma visão mais sistemática das principais informações do negócio. Em outras palavras, é um recurso que visa consolidar os dados de maior relevância em um painel, facilitando o processo de análise e a tomada de decisão.

\par O uso de planilhas e relatórios já são ultrapassados para análises de dados, não sendo suficientes para suprir as necessidades mais urgentes. Conforme a tecnologia foi evoluindo no mundo corporativo, surgiram os dashboards que evitam esforços desnecessários e ter uma visão mais ampla de todo o cenário corporativo para, assim, tomar decisões estratégicas e assertivas.

\par A visualização de dados através de dashboards já é uma realidade em softwares de gestão empresarial, integrando painéis de controle com inteligência artificial e fornecendo informações atualizadas automaticamente. É possível personalizá-los e comparar dados através de filtros, que facilitam análises de indicadores.


\subsection{Uma ferramenta de apoio à decisão}

\par Segundo um estudo feito pela UNINDU (The International Congress on University Industry), 83% das pessoas absorvem melhor as informações através da visão. Isso demonstra a importância do dashboard para tomada de decisões rápidas e melhor análise dos indicadores,melhorando metas e atingindo objetivos.

\par O Business Intelligence (BI), por exemplo, é uma área que exige precisão na coleta e controle das informações para gerar insights em base desta ferramenta. Os dados são agrupados em conjuntos de registros e disponibilizadas por meio de dashboards para mensurar o desempenho atual e futuro da empresa de acordo com o cenário.

\par Além disso, os dashboards monitoram os dados, com o intuito de melhorar todos os processos. Eles permitem que o usuário personalize painéis e filtre informações para a visualização dos resultados, como quantidade, tempo e outras opções.

\subsection{Benefícios do uso para as empresas}

A aplicação dos dashboards na gestão empresarial traz muitos benefícios para a tomada de decisão e a visão estratégica do seu negócio. 

\subsubsection{Auxilia na tomada de decisões}
\par O processo de tomada de decisões fica cada vez mais fácil através dos dashboards, que centralizam informações de fácil visualização e compreensão, possibilitando uma visão ampla do seu negócio. 

\subsubsection{Transparência de informações}
\par Na gestão, é importante que todas as equipes tenham acesso aos indicadores da empresa, mantendo a transparência das informações. As ferramentas de dashboard tem o objetivo de facilitar a comunicação interna entre todos os profissionais.

\subsubsection{Otimização de tempo e recursos}
\par A visualização por dashboards otimiza o tempo para tomarem decisões e evitarem trabalhos manuais e complexos com a organização de dados, passando a priorizar outras atividades mais relevantes.

\subsubsection{Alinhamento estratégico}
\par Com as informações consolidadas em um único painel, a gestão se torna mais ágil e efetiva, possibilitando o alinhamento de estratégias e decisões para o negócio.

\subsection{Antecipação de problemas}

\par Como o dashboard trabalha com atualizações constantes e análises mais específicas, é mais fácil prever problemas e tendências negativas que podem vir a acontecer. Estes problemas ficam mais explícitos com o uso da tecnologia de inteligência artificial que os identifica de maneira mais fácil.

\par Qualquer mudança é detectada com mais simplicidade e o tempo para pensar nas possíveis soluções se torna muito maior. Isso melhora o processo de tomada de decisão e evita possíveis prejuízos. 

\subsection{Experiência de uso}

\par Uma boa interface e uma boa experiência de uso se dá pela arquitetura das informação do dashboard. É fundamental que seja organizado, coerente e intuitivo. O objetivo é tornar o mais fácil possível encontrar o que se procura. Através de menus, cores e  símbolos é possível saber quais são as opções e deixar claro as consequências que cada ação irá gerar. Dessa forma, a experiência do usuário ao usar o dashboard será rápida e efetiva, atendendo suas expectativas.

\section{Assunto 2}

XXXXXX
\\\\
XXXXXX




\section{Assunto 3}

XXXXXX
\\\\
XXXXXX

\subsection{SubAssunto 3}

XXXXXX
\\\\
XXXXXX

\section{Assunto 4}

XXXXXX
\\\\
XXXXXX

\subsection{SubAssunto 4}

XXXXXX
\\\\
XXXXXX


