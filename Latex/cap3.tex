\chapter{Metodologia}
\label{Cap:MateriaisMetodos}

% Capítulo 3: Materiais e Métodos (ou Metodologia)
\indent
\par Este capítulo aborda os materiais e métodos utilizados no projeto, contendo maiores detalhes sobre algoritmos, tecnologias e estratégias empregadas na solução.

\section{Modelo da solução}
\indent
\par A fim de solucionar o problema identificado, foram definidos métodos e estratégias com o objetivo de criar um fluxo de trabalho. As Figura 1 e 2 fornecem um diagrama ilustrativo e simplificado da solução proposta.
\subsection{Diagramas da solução}

\begin{figure}[H]
    \centering
    \caption{Diagrama de blocos da solução}
    \includegraphics[width=1.0\linewidth]{Imagens/diagramaDeBlocos.jpg}
    \caption*{Fonte: Arquivo dos autores (2020)}
    \label{autoai-results}
\end{figure}

\begin{figure}[H]
    \centering
    \caption{Ilustração da solução}
    \includegraphics[width=1.0\linewidth]{Imagens/DiagramaDeBlocosIcones.png}
    \caption*{Fonte: Arquivo dos autores (2020)}
    \label{autoai-results}
\end{figure}

\section{Python}
\indent
\par Linguagem de programação de alto nível lançada em 1991. Atualmente possui um modelo de desenvolvimento open source e gerenciado pela Python Software Foundation. A linguagem prioriza a legibilidade de código e possui poderosos recursos advindos de suas bibliotecas padrão combinados com bibliotecas de terceiros.

\section{Pandas}
\indent
\par Biblioteca desenvolvida em Python que possui estruturas de dados que facilitam e agilizam a manipulação e análise de dados.

\section{Numpy}
\indent
\par Biblioteca desenvolvida em Python criada para facilitar o desenvolvimento de aplicações com fins matemáticos e complexidade computacional avançada. Possui suporte para vetores e matrizes multidimensionais e diversas funções matemáticas para interação com essas estruturas.

\section{OpenCV}
\indent
\par Se trata de uma iniciativa open source que teve sua primeira versão lançada nos anos 2000 e continua em expansão até os dias atuais. Atualmente, suporta uma ampla variedade de algoritmos relacionados a Visão Computacional e Machine Learning, além de estar disponível em diversas linguagens de programação como C++, Python, Java e diferentes plataformas como Windows, Linux, OS X, Android e IOS.

\section{Redes Neurais}
\indent
\par Sistemas de computação que tem como objetivo reconhecer e classificar padrões em dados brutos. Tais sistemas buscam agir como o sistema nervoso humano, aprendendo e melhorando continuamente.

\section{YOLO}
\indent
\par Método para detecção e classificação de objetos em uma imagem combinando OpenCV e redes neurais. Se tornou popular pela sua grande eficiência e agilidade quando comparado com outros frameworks desenvolvidos anteriormente.

\section{Django}
\indent
\par Framework de alto nível, gratuito e open source desenvolvido em Python para programação de aplicações web. Apoia o desenvolvimento rápido e limpo, possuindo muitas ferramentas e métodos previamente construídos para facilitar e apoiar o desenvolvedor na criação das aplicações.

\section{Power BI}
\indent
\par Serviço de análise de negócios lançado em 2015 pela Microsoft. Permite a criação e compartilhamento de dashboards para a visualização de dados de forma interativa, além de possuir integração com outras ferramentas, como o Microsoft Excel.


