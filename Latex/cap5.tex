\chapter{Conclusão e Trabalhos futuros}
\label{Cap:Conclusoes}

% Capítulo 5: Conclusões
\section{Conclusão}
\section{Trabalhos Futuros}
\indent
\par Tendo em vista que a aplicação salva dados diariamente, a quantidade de informações armazenadas tende a crescer rapidamente. Com isso, o espaço em disco, memória e CPU necessários na instância EC2 que armazena o banco de dados precisaria aumentar paralelamente aos dados. Por esse motivo, seria válido que o banco de dados ficasse em uma instância apartada da instância de aplicação, garantindo maior escalabilidade e persistência dos dados. Esse objetivo poderia ser alcançado utilizando o serviço RDS da AWS.
\indent
\par Juntamente com a criação de uma nova instância para o banco de dados, considerando que a aplicação já esteja armazenando informações há um longo período, também seria viável um estudo mais aprofundado das informações e a aplicação de técnicas de inteligência artificial, por meio das quais seria possível a extração de informações como a previsão de velocidade e atrasos de uma via, previsão de frota, tempo, trânsito etc. Essas e outras informações poderiam ser obtidas, analisadas e incluídas no painel.
\indent
\par Por fim, outra informação que agregaria valor no projeto seria a lotação dos ônibus, que poderia ser extraída por meio de análise de imagens de câmeras já existentes, instaladas dentro dos veículos. Essas imagens passariam por um tratamento e treinamento em uma rede artificial que seria utilizada pelo yolo, que por sua vez identificaria a quantidade de pessoas presentes dentro de cada veículo, classificando cada ônibus como “vazio”, “normal” ou “cheio”.




