\def\Ano{\the\year}

\titulo{Inteligência para Transporte Público}
\local{São Caetano do Sul}
\data{\Ano}
\instituicao{Escola de Engenharia Mauá}
\orientador{Tiago Sanches da Silva}
\coorientador{Murilo Zanini de Carvalho}
%{Murilo Zanini de Carvalho}

\tipotrabalho{Trabalho de Conclusão de Curso}
\newcommand{\IMT}{Instituto Mauá de Tecnologia}
\newcommand{\CEUN}{Centro Universitário}
\newcommand{\EscolaCompleto}{\imprimirinstituicao~do~\CEUN~do~\IMT}
\newcommand{\TitutoGraduacao}{Engenheiro de Computação}
\newcommand{\Curso}{Engenharia de Computação}
\newcommand{\DataBanca}{29 de Abril de \Ano}
\newcommand{\Estado}{SP}




% palavras-chave
\newcommand{\PalavraChaveA}{}
\newcommand{\PalavraChaveB}{painel de controle}
\newcommand{\PalavraChaveC}{inteligência artificial}
\newcommand{\PalavraChaveD}{análise de dados}
\newcommand{\PalavraChaveE}{decisão}
\newcommand{\PalavraChaveF}{transporte público}

% key-words
\newcommand{\KeyWordA}{}
\newcommand{\KeyWordB}{dashboard}
\newcommand{\KeyWordC}{artificial intelligence}
\newcommand{\KeyWordD}{data analysis}
\newcommand{\KeyWordE}{decision}
\newcommand{\KeyWordF}{public transportation}

% autores
\newcommand{\FulanoASnome}{Novello}
\newcommand{\FulanoBSnome}{Miraglia}
\newcommand{\FulanoCSnome}{Araujo}
\newcommand{\FulanoDSnome}{}

\newcommand{\AFulano}{\FulanoASnome, Arthur Segura}
\newcommand{\BFulano}{\FulanoBSnome, Luca Ezellner}
\newcommand{\CFulano}{\FulanoCSnome, Lucas Marques}
\newcommand{\DFulano}{}

\newcommand{\FulanoA}{ Arthur Segura Ortiz Novello }
\newcommand{\FulanoB}{ Luca Ezellner Miraglia }
\newcommand{\FulanoC}{ Lucas Marques de Araujo }
\newcommand{\FulanoD}{}

\autor{\FulanoA \\ \FulanoB \\ \FulanoC \\ \FulanoD}




% Banca Examinadora
\newcommand{\Tratamento}[2]{
Prof\ifthenelse{\equal{#2}{m}}{}{$^{\underline{a}}$}.~\ifthenelse{\equal{#1}{Me}}{\ifthenelse{\equal{#2}{m}}{Me.}{Ma.}}{\ifthenelse{\equal{#1}{Esp} \OR \equal{#1}{Dr}}{#1\ifthenelse{\equal{#2}{m}}{. }{$^{\underline{a}}$. }}}
}

%Nome completo dos Examinadores
\newcommand{\ExaminadorA}{Murilo Zanini de Carvalho}
\newcommand{\ExaminadorB}{Sergio Ribeiro Augusto}

%Instituições dos Examinadores
\newcommand{\InstituicaoExaminadorA}{\imprimirinstituicao}
\newcommand{\InstituicaoExaminadorB}{\imprimirinstituicao}

%Título dos Examinadores (Me, Esp ou Dr)
\newcommand{\SexoOrientador}{m}
\newcommand{\SexoCoorientador}{m}
\newcommand{\SexoExaminadorA}{m}
\newcommand{\SexoExaminadorB}{m}

\newcommand{\TituloOrientador}{Me}
\newcommand{\TituloCoorientador}{}
\newcommand{\TituloExaminadorA}{Me}
\newcommand{\TituloExaminadorB}{Dr}

\newcommand{\TratamentoOrientador}  {\Tratamento{\TituloOrientador}{\SexoOrientador}}
\newcommand{\TratamentoCoorientador}{\Tratamento{\TituloCoorientador}{\SexoCoorientador}}
\newcommand{\TratamentoExaminadorA} {\Tratamento{\TituloExaminadorA}{\SexoExaminadorA}}
\newcommand{\TratamentoExaminadorB} {\Tratamento{\TituloExaminadorB}{\SexoExaminadorB}}

% alterando valores do ABNTEX2
\renewcommand{\orientadorname}{\ifthenelse{\equal{\SexoOrientador}{m}}{Orientador}{Orientadora}}
\renewcommand{\coorientadorname}{\ifthenelse{\equal{\SexoCoorientador}{m}}{Coorientador}{Coorientadora}}





% Informações de dados para FOLHA DE ROSTO e FOLHA DE APROVAÇÃO
\preambulo{\imprimirtipotrabalho~apresentado à~\EscolaCompleto~como requisito parcial para a obtenção de título de~\TitutoGraduacao. \newline Área de Concentração:~\Curso}

\newcommand{\preambuloaprovado}{\imprimirtipotrabalho~aprovado como requisito parcial para a obtenção de título de~\TitutoGraduacao~pela \EscolaCompleto. \newline Área de Concentração:~\Curso}





%Ficha Catalográfica
%CDU - Classificação Decimal Universal
%6 - Ciências Aplicadas. Medicina. Tecnologia.
%2 - Engenharia. Tecnologia geral.
%1 - Engenharia mecânica em geral. Tecnologia nuclear. Engenharia elétrica. Maquinaria
%3 - Engenharia elétrica
\newcommand{\CDU}{CDU 621.398}
\newcommand{\Cutter}{T272}





% tamanho da fonte das seções
\renewcommand{\ABNTEXchapterfontsize}{\normalsize}
\renewcommand{\ABNTEXsectionfontsize}{\normalsize}
\renewcommand{\ABNTEXsubsectionfontsize}{\normalsize}
\renewcommand{\ABNTEXsubsubsectionfontsize}{\normalsize}
\renewcommand{\ABNTEXsubsubsubsectionfontsize}{\normalsize}
% tipo da fonte das seções
\renewcommand{\ABNTEXchapterfont}{\normalfont\bfseries}
\renewcommand{\ABNTEXsectionfont}{\normalfont\bfseries}
\renewcommand{\ABNTEXsubsectionfont}{\normalfont\bfseries}
\renewcommand{\ABNTEXsubsubsectionfont}{\normalfont\bfseries}
\renewcommand{\ABNTEXsubsubsubsectionfont}{\normalfont\bfseries}
\renewcommand{\legend}{\ABNTEXfontereduzida\centering}

% Espaçamentos 
% Margens
\setlrmarginsandblock{3cm}{2cm}{*}
\setulmarginsandblock{3cm}{2cm}{*}
\checkandfixthelayout
\setlength\afterchapskip{\onelineskip}
\setlength{\parindent}{0 mm} % recuo




% Criando Quadros
\newcounter{quadroCount}
\setcounter{quadroCount}{0}
\newenvironment{quadro}[2] % #1 caption e #2 label
{% This is the begin code
  \refstepcounter{quadroCount}
  \centering
  {Quadro} \arabic{quadroCount}: #2\\ 
	\label{#1}
}
{% This is the end code
}