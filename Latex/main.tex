% ------------------------------------------------------------------------
% Trabalho de Conclusão de Curso CEUN-IMT
% ------------------------------------------------------------------------

\title{TCC - Inteligência para transporte público}

\documentclass[
	% -- opções da classe memoir --
	12pt,							% tamanho da fonte
	a4paper,					% tamanho do papel. 
	openright,				% capítulos começam em pág ímpar (insere página vazia caso preciso)
	oneside,					% só frente 
	%ou twoside,   		% frente e verso
	% -- opções do pacote babel --
	english,					% idioma adicional para hifenização
	brazil						% o último idioma é o principal do documento
	]{abntex2}

%% pacotes utilizados
\usepackage[brazil]{babel}
\usepackage{longtable}
\usepackage{tabularx}
\usepackage{float}
\usepackage[table]{xcolor}

\usepackage[utf8]{inputenc}			% Codificacao do documento (conversão automática dos acentos)
\usepackage[T1]{fontenc}				% Selecao de codigos de fonte.
\usepackage{times}							% Usa a fonte Times	
\usepackage{latexsym}
\usepackage{amsmath}
\usepackage{lastpage}						% Usado pela Ficha catalográfica
\usepackage{fancybox}
%Para não definir footnote
\let\footruleskip\undefined
\usepackage{fancyhdr}
\usepackage{lscape}
\usepackage{setspace}
\usepackage{textcomp}						% usar o TM  \texttrademark
\usepackage[pdftex]{graphicx}
\usepackage{subcaption}
\setlength{\belowcaptionskip}{15pt plus 3pt minus 2pt} % Chosen fairly arbitrarily
\usepackage{color}							% Controle das cores
\usepackage{microtype}					% para melhorias de justificação
\usepackage{multirow}
\usepackage{float}
\usepackage{pdfpages}						% incluir outro pdf como anexo
\usepackage{hyperref} 					% controla a formação do índice
\usepackage{booktabs}						% tabelas com linhas variáveis
\usepackage{dirtree}						% árvore de diretórios do apêndice com conteúdo do CD
\usepackage{url}

% Configurações
\def\Ano{\the\year}

\titulo{Inteligência para Transporte Público}
\local{São Caetano do Sul}
\data{\Ano}
\instituicao{Escola de Engenharia Mauá}
\orientador{Tiago Sanches da Silva}
\coorientador{}
%{Murilo Zanini de Carvalho}

\tipotrabalho{Trabalho de Conclusão de Curso}
\newcommand{\IMT}{Instituto Mauá de Tecnologia}
\newcommand{\CEUN}{Centro Universitário}
\newcommand{\EscolaCompleto}{\imprimirinstituicao~do~\CEUN~do~\IMT}
\newcommand{\TitutoGraduacao}{Engenheiro de Computação}
\newcommand{\Curso}{Engenharia de Computação}
\newcommand{\DataBanca}{29 de Abril de \Ano}
\newcommand{\Estado}{SP}




% palavras-chave
\newcommand{\PalavraChaveA}{xxxxxxx}
\newcommand{\PalavraChaveB}{xxxxxxx}
\newcommand{\PalavraChaveC}{xxxxxxx}
\newcommand{\PalavraChaveD}{xxxxxxx}
\newcommand{\PalavraChaveE}{xxxxxxx}
\newcommand{\PalavraChaveF}{xxxxxxx}

% key-words
\newcommand{\KeyWordA}{xxxxxxx}
\newcommand{\KeyWordB}{xxxxxxx}
\newcommand{\KeyWordC}{xxxxxxx}
\newcommand{\KeyWordD}{xxxxxxx}
\newcommand{\KeyWordE}{xxxxxxx}
\newcommand{\KeyWordF}{xxxxxxx}

% autores
\newcommand{\FulanoASnome}{Novello}
\newcommand{\FulanoBSnome}{Miraglia}
\newcommand{\FulanoCSnome}{Araujo}
\newcommand{\FulanoDSnome}{}

\newcommand{\AFulano}{\FulanoASnome, Arthur Segura}
\newcommand{\BFulano}{\FulanoBSnome, Luca Ezellner}
\newcommand{\CFulano}{\FulanoCSnome, Lucas Marques}
\newcommand{\DFulano}{}

\newcommand{\FulanoA}{ Arthur Segura Ortiz Novello }
\newcommand{\FulanoB}{ Luca Ezellner Miraglia }
\newcommand{\FulanoC}{ Lucas Marques de Araujo }
\newcommand{\FulanoD}{}

\autor{\FulanoA \\ \FulanoB \\ \FulanoC \\ \FulanoD}




% Banca Examinadora
\newcommand{\Tratamento}[2]{
Prof\ifthenelse{\equal{#2}{m}}{}{$^{\underline{a}}$}.~\ifthenelse{\equal{#1}{Me}}{\ifthenelse{\equal{#2}{m}}{Me.}{Ma.}}{\ifthenelse{\equal{#1}{Esp} \OR \equal{#1}{Dr}}{#1\ifthenelse{\equal{#2}{m}}{. }{$^{\underline{a}}$. }}}
}

%Nome completo dos Examinadores
\newcommand{\ExaminadorA}{Murilo Zanini de Carvalho}
\newcommand{\ExaminadorB}{Sergio Ribeiro Augusto}

%Instituições dos Examinadores
\newcommand{\InstituicaoExaminadorA}{\imprimirinstituicao}
\newcommand{\InstituicaoExaminadorB}{\imprimirinstituicao}

%Título dos Examinadores (Me, Esp ou Dr)
\newcommand{\SexoOrientador}{m}
\newcommand{\SexoCoorientador}{m}
\newcommand{\SexoExaminadorA}{m}
\newcommand{\SexoExaminadorB}{m}

\newcommand{\TituloOrientador}{Me}
\newcommand{\TituloCoorientador}{}
\newcommand{\TituloExaminadorA}{Me}
\newcommand{\TituloExaminadorB}{Dr}

\newcommand{\TratamentoOrientador}  {\Tratamento{\TituloOrientador}{\SexoOrientador}}
\newcommand{\TratamentoCoorientador}{\Tratamento{\TituloCoorientador}{\SexoCoorientador}}
\newcommand{\TratamentoExaminadorA} {\Tratamento{\TituloExaminadorA}{\SexoExaminadorA}}
\newcommand{\TratamentoExaminadorB} {\Tratamento{\TituloExaminadorB}{\SexoExaminadorB}}

% alterando valores do ABNTEX2
\renewcommand{\orientadorname}{\ifthenelse{\equal{\SexoOrientador}{m}}{Orientador}{Orientadora}}
\renewcommand{\coorientadorname}{\ifthenelse{\equal{\SexoCoorientador}{m}}{Coorientador}{Coorientadora}}





% Informações de dados para FOLHA DE ROSTO e FOLHA DE APROVAÇÃO
\preambulo{\imprimirtipotrabalho~apresentado à~\EscolaCompleto~como requisito parcial para a obtenção de título de~\TitutoGraduacao. \newline Área de Concentração:~\Curso}

\newcommand{\preambuloaprovado}{\imprimirtipotrabalho~aprovado como requisito parcial para a obtenção de título de~\TitutoGraduacao~pela \EscolaCompleto. \newline Área de Concentração:~\Curso}





%Ficha Catalográfica
%CDU - Classificação Decimal Universal
%6 - Ciências Aplicadas. Medicina. Tecnologia.
%2 - Engenharia. Tecnologia geral.
%1 - Engenharia mecânica em geral. Tecnologia nuclear. Engenharia elétrica. Maquinaria
%3 - Engenharia elétrica
\newcommand{\CDU}{CDU 621.398}
\newcommand{\Cutter}{T272}





% tamanho da fonte das seções
\renewcommand{\ABNTEXchapterfontsize}{\normalsize}
\renewcommand{\ABNTEXsectionfontsize}{\normalsize}
\renewcommand{\ABNTEXsubsectionfontsize}{\normalsize}
\renewcommand{\ABNTEXsubsubsectionfontsize}{\normalsize}
\renewcommand{\ABNTEXsubsubsubsectionfontsize}{\normalsize}
% tipo da fonte das seções
\renewcommand{\ABNTEXchapterfont}{\normalfont\bfseries}
\renewcommand{\ABNTEXsectionfont}{\normalfont\bfseries}
\renewcommand{\ABNTEXsubsectionfont}{\normalfont\bfseries}
\renewcommand{\ABNTEXsubsubsectionfont}{\normalfont\bfseries}
\renewcommand{\ABNTEXsubsubsubsectionfont}{\normalfont\bfseries}
\renewcommand{\legend}{\ABNTEXfontereduzida\centering}

% Espaçamentos 
% Margens
\setlrmarginsandblock{3cm}{2cm}{*}
\setulmarginsandblock{3cm}{2cm}{*}
\checkandfixthelayout
\setlength\afterchapskip{\onelineskip}
\setlength{\parindent}{0 mm} % recuo




% Criando Quadros
\newcounter{quadroCount}
\setcounter{quadroCount}{0}
\newenvironment{quadro}[2] % #1 caption e #2 label
{% This is the begin code
  \refstepcounter{quadroCount}
  \centering
  {Quadro} \arabic{quadroCount}: #2\\ 
	\label{#1}
}
{% This is the end code
}

% Bibliografia
\usepackage[brazilian,hyperpageref]{backref}		% Paginas com as citações na bibl
\usepackage[
  alf,
	abnt-emphasize=bf,
	abnt-url-package=hyperref,
	abnt-etal-list=0, % 6023 - mostra todos os nomes nas referências
	abnt-etal-cite=3, % 10520 - número de autores antes de ser et al
	recuo=0.0cm %0.5cm
]{abntex2cite}										% Citações padrão ABNT


% Configurações o PDF 
\makeatletter
	\hypersetup{
	pdftitle={\imprimirtitulo}, 
	pdfauthor={{\imprimirautor~e~\imprimirorientador}},
	pdfsubject={\imprimirpreambulo},
	pdfkeywords={\PalavraChaveA\\}{\PalavraChaveB\\}{\PalavraChaveC\\}{\PalavraChaveD\\}{\PalavraChaveE},
	colorlinks=true,						% não desenha o retângulo em referências
	linkcolor=black, 						% cor das referências cruzadas
	citecolor=black, 						% cor das citações - 10520
	filecolor=black, 						% cor dos arquivos
	urlcolor=black, 						% cor dos sites
	bookmarksdepth=4
}
\makeatother



% Início do documento
\begin{document}


		% Elementos pré-textuais
    \pretextual
		\renewcommand{\imprimircapa}{%
	\begin{capa}%
		\center
		\textbf{\imprimirautor}

		\vspace*{\fill}

		\textbf{\imprimirtitulo}

		\vspace*{\fill}
		    
		{\imprimirlocal}
		\par
		{\imprimirdata}
	\end{capa}
}
\imprimircapa
		\makeatletter
\renewcommand{\folhaderostocontent}{
	\begin{center}
		{\ABNTEXchapterfont \textbf{\imprimirautor}}

		\vspace*{\fill}\vspace*{\fill}

		\ABNTEXchapterfont \textbf{\imprimirtitulo}

		\vspace*{\fill}
		
		\normalfont

		\begin{flushright}
			\abntex@ifnotempty{\imprimirpreambulo}{
				\hspace{.45\textwidth}
	
				\begin{minipage}{.55\textwidth}
					\SingleSpacing
					\imprimirpreambulo
					
					\vspace*{.7cm}

					{\imprimirorientadorRotulo~\imprimirorientador\par}
					\abntex@ifnotempty{\imprimircoorientador}{
						{\imprimircoorientadorRotulo~\imprimircoorientador}
					}
				\end{minipage}
	
				\vspace*{\fill}
			}
		\end{flushright}

		\vspace*{\fill}

		{\imprimirlocal}
		\par
		{\imprimirdata}
	\end{center}
}
\makeatother

\imprimirfolhaderosto*
%


% deixar essas duas linhas de espaço!!!!			
		\begin{fichacatalografica}
	%\includepdf{Ficha_catalografica.pdf}
% ou

\newcolumntype{L}[1]{>{\raggedright\let\newline\\\arraybackslash\hspace{0pt}}p{#1}}
\newcounter{cont}
\setcounter{cont}{1}

\newif\ifnAlunosTresMais
\ifthenelse{\equal{\FulanoDSnome}{}}{\nAlunosTresMaisfalse}{\nAlunosTresMaistrue}

%\newif\ifnAlunosTresMais
%\ifthenelse{\equal{\FulanoDSnome}{}}{\nAlunosTresMaisfalse}{\nAlunosTresMaistrue}

\vspace*{\fill}
\ifnAlunosTresMais
	\begin{table}[!b]
  		\setlength{\arrayrulewidth}{0.75pt}
		\begin{tabular}{|L{1cm}L{11.5cm}|}
			\hline
			~ & ~ \\
			{\fontsize{9pt}{9pt}\selectfont \Cutter} & 
			{\fontsize{9pt}{9pt}\selectfont \hspace{0.37cm}
			\imprimirtitulo\ / \FulanoA...[et al.] - \local {} : CEUN-IMT, \imprimirdata.}\\
			~ & \hspace{0.37cm}
			{\fontsize{9pt}{9pt}\selectfont \pageref{LastPage} p.}\\
			~ & ~ \\
			~ & \hspace{0.37cm}
			{\fontsize{9pt}{1cm}\selectfont \imprimirtipotrabalho~- \EscolaCompleto, \local,~\Estado,~\imprimirdata.}\\
			~ & ~ \\
			~ & ~ \\
			~ & \hspace{0.37cm}
			{\fontsize{9pt}{1cm}\selectfont \imprimirorientadorRotulo:~\imprimirorientador}\\	
			~ & ~ \\
			~ & \hspace{0.37cm}
			{\fontsize{9pt}{1cm}\selectfont 
			\ifthenelse{\equal{\PalavraChaveA}{}}{}{1. \PalavraChaveA} 
			\ifthenelse{\equal{\PalavraChaveB}{}}{}{2. \PalavraChaveB} 
			\ifthenelse{\equal{\PalavraChaveC}{}}{}{3. \PalavraChaveC} 
			\ifthenelse{\equal{\PalavraChaveD}{}}{}{4. \PalavraChaveD} 
			\ifthenelse{\equal{\PalavraChaveE}{}}{}{5. \PalavraChaveE} 
     	\ifthenelse{\equal{\FulanoASnome}{}}{}{\Roman{cont}. \AFulano.\setcounter{cont}{\value{cont}+1}} 
     	\ifthenelse{\equal{\FulanoBSnome}{}}{}{\Roman{cont}. \BFulano.\setcounter{cont}{\value{cont}+1}} 
     	\ifthenelse{\equal{\FulanoCSnome}{}}{}{\Roman{cont}. \CFulano.\setcounter{cont}{\value{cont}+1}} 
     	\ifthenelse{\equal{\FulanoDSnome}{}}{}{\Roman{cont}. \DFulano.\setcounter{cont}{\value{cont}+1}} 
     	\Roman{cont}. \IMT. \CEUN. \imprimirinstituicao. \setcounter{cont}{\value{cont}+1}
     	\Roman{cont}. Tí­tulo.}	\\
			~ & ~ \\
			~ & \hspace{9cm}
			{\fontsize{9pt}{1cm}\selectfont \CDU} \\
 	 		\hline
  		\end{tabular}
	\end{table}
\else
	\begin{table}[!b]
	\centering
  		\setlength{\arrayrulewidth}{0.75pt}
		\begin{tabular}{|L{1cm}L{11.5cm}|}
			\hline
			~ & ~ \\
% 			{\fontsize{9pt}{9pt}\selectfont \Cutter} 
            & 
			{\fontsize{9pt}{9pt}\selectfont \AFulano}\\
			~ & {\fontsize{9pt}{9pt}\selectfont \hspace{0.37cm}
			\imprimirtitulo\ / \ifthenelse{\equal{\FulanoASnome}{}}{}{\FulanoA}\ifthenelse{\equal{\FulanoBSnome}{}}{}{, \FulanoB}\ifthenelse{\equal{\FulanoCSnome}{}}{}{, \FulanoC}. - \imprimirlocal: CEUN-IMT, \imprimirdata.}\\
			~ & {\fontsize{9pt}{9pt}\selectfont \hspace{0.37cm} \pageref{LastPage} p.}\\
			~ & ~ \\
			~ & \hspace{0.37cm}
			{\fontsize{9pt}{1cm}\selectfont \imprimirtipotrabalho~-~\EscolaCompleto, \imprimirlocal, \Estado, \imprimirdata.}\\
			~ & ~ \\
			~ & ~ \\
			~ & \hspace{0.37cm}
			{\fontsize{9pt}{1cm}\selectfont \imprimirorientadorRotulo:~Prof. Me.~\imprimirorientador}\\	
			~ & ~ \\
			~ & \hspace{0.37cm}
			{\fontsize{9pt}{1cm}\selectfont 
			\ifthenelse{\equal{\PalavraChaveA}{}}{}{1. \PalavraChaveA}. 
			\ifthenelse{\equal{\PalavraChaveB}{}}{}{2. \PalavraChaveB}. 
			\ifthenelse{\equal{\PalavraChaveC}{}}{}{3. \PalavraChaveC}. 
			\ifthenelse{\equal{\PalavraChaveD}{}}{}{4. \PalavraChaveD}. 
     	\ifthenelse{\equal{\FulanoBSnome}{}}{}{\Roman{cont}. \BFulano.\setcounter{cont}{\value{cont}+1}} 
     	\ifthenelse{\equal{\FulanoCSnome}{}}{}{\Roman{cont}. \CFulano.\setcounter{cont}{\value{cont}+1}} 
     	\Roman{cont}. \IMT. \imprimirinstituicao. \setcounter{cont}{\value{cont}+1}
     	\Roman{cont}. Tí­tulo.}	\\
			~ & ~ \\
% 			~ & \hspace{9cm}
% 			{\fontsize{9pt}{1cm}\selectfont \CDU} \\			
 	 		\hline
  		\end{tabular}
	\end{table}
\fi

\end{fichacatalografica}

		\makeatletter
\begin{folhadeaprovacao}
	\begin{center}
		{\ABNTEXchapterfont \textbf{\imprimirautor}}
		
		\vspace*{\fill}\vspace*{\fill}
		
		\begin{center}
			\ABNTEXchapterfont \textbf{\imprimirtitulo}
		\end{center}

		\vspace*{\fill}
		\hspace{.45\textwidth}

		\normalfont

		\begin{flushright}
			\begin{minipage}{.55\textwidth}
				\preambuloaprovado
			\end{minipage}%
		\end{flushright}

		\vspace*{\fill}
	\end{center}
		
	\normalfont

	Banca examinadora:

	\begin{center}
		\TratamentoOrientador~\imprimirorientador \\ \imprimirorientadorRotulo \\
		\vspace*{\fill}
		\abntex@ifnotempty{\imprimircoorientador}{
			\TratamentoCoorientador~\imprimircoorientador \\ \imprimircoorientadorRotulo \\ \imprimirinstituicao\\
			\vspace*{\fill}
		}
		\TratamentoExaminadorA~\ExaminadorA \\ Avaliador\\
		\vspace*{\fill}
		\TratamentoExaminadorB~\ExaminadorB \\ Avaliador\\
		\vspace*{\fill}
	\end{center}
	
	\begin{center}
		\vspace*{0.5cm}

		{\imprimirlocal}, \DataBanca.
	\end{center}
\end{folhadeaprovacao}
\makeatother
		
		\begin{dedicatoria}
    \vspace*{\fill}
	\begin{flushright}
		\textit{Aos nossos pais, irmãos e irmãs, amigos e colegas de formação.}
	\end{flushright}
\end{dedicatoria}

		\begin{agradecimentos}
XXXXyyy
\\\\
XXXXX
\\\\
XXXXX
\\\\
XXXXX
\\\\
XXXXX
\end{agradecimentos}

		\begin{epigrafe}
    \vspace*{\fill}
	\begin{flushright}
		\textit{``XX''} \\("XX")\\
	\textit{\bfseries XX}
	\end{flushright}
\end{epigrafe}

		\begin{resumo}
XXXXXXXXXXXXX
\\\\
\textbf{Palavras-chaves}: \PalavraChaveA.~\ifthenelse{\equal{\PalavraChaveB}{}}{}{\PalavraChaveB.~}\ifthenelse{\equal{\PalavraChaveC}{}}{}{\PalavraChaveC.~}\ifthenelse{\equal{\PalavraChaveD}{}}{}{\PalavraChaveD.~}\ifthenelse{\equal{\PalavraChaveE}{}}{}{\PalavraChaveE.~}\ifthenelse{\equal{\PalavraChaveF}{}}{}{\PalavraChaveF.~}

\end{resumo}

		\begin{resumo}[Abstract]
	\begin{otherlanguage*}{english}
	\indent
	\par This project aims to create a dashboard that monitors the demand of vehicles in bus lines of SPTrans, the company responsible for buses in the city of São Paulo. For this monitoring, public data provided by SPTrans was used, such as bus routes, GPS data, lines and stops, for example, which will be enriched with other information, like dates and locations of events, status of train and subway lines, weather information and traffic around the city.
	\par Although a lot of public transport data is public available, there isn’t a system to aggregate all this information that would be able to help people who use this kind of transport on a daily basis with a overview about the lines condition. The present work intends to contribute positively to improve the user experience and bring a more transparent and complete view of the service.
	\par During the development, tools such as Python, Django, Django-REST, Celery and Amazon Web Services (AWS) were used, which made it possible to create a dashboard and all its features, from the acquisition of external data, to the manipulation of that data and deployment of the platform in a cloud platform.
	\\
	\\
	\textbf{Key-words}: ~\ifthenelse{\equal{\KeyWordB}{}}{}{\KeyWordB.~}\ifthenelse{\equal{\KeyWordC}{}}{}{\KeyWordC.~}\ifthenelse{\equal{\KeyWordD}{}}{}{\KeyWordD.~}\ifthenelse{\equal{\KeyWordE}{}}{}{\KeyWordE.~}\ifthenelse{\equal{\KeyWordF}{}}{}{\KeyWordF.~}
	\end{otherlanguage*}
\end{resumo}

		\listoffigures*
			\cleardoublepage
		\listoftables*
			\cleardoublepage
		\begin{siglas}
    \item[API] \textit{Application Programming Interface}
	\item[IPEA] {Instituto de Pesquisa Econômica Aplicada}
	\item[CPTM] {Companhia Paulista de Trens Metropolitanos}
	\item[IBGE] {Instituto Brasileiro de Geografia e Estatística}
	\item[IA]  {Inteligência Artificial}
	\item[SPTrans] {São Paulo Transportes}
	\item[AWS] \textit{Amazon Web Services}
	\item[EC2] \textit{Elastic Compute Cloud}
	\item[SQS] \textit{Simple Queue Service}
	\item[CRUD] \textit{Create, Read, Update, Delete}
	\item[REST] \textit{Representational State Transfer}
	\item[IoT] \textit{Internet of Things}
	\item[GPS] \textit{Global Positional System}
	\item[CCTV] \textit{Closed Circuit Television}
	\item[IEEE] {Instituto de Engenheiros Eletricistas e Eletrônicos}
	\item[OpenCV] \textit{Open Source Computer Vision}
	\item[GTFS] \textit{General Transit Feed Specification}
	\item[UNINDU] \textit{The International Congress on University Industry}
	\item[BI] \textit{Business Inteligence}
	\item[PIB] {Produto Interno Bruto}
	\item[CAGR] {Taxa Anual Composta de Crescimento}
	\item[UFSC] {Universidade Federal de Santa Catarina}
	\item[Relu] \textit{Rectified Linear Unity}
	\item[Elu] \textit{Exponential Linear Unity}
	\item[BMTC] \textit{Bengaluru Metropolitan Transport Corporation}
	\item[IISC] {Instituto Indiano de Ciência}
	\item[noSQL] \textit{Not Only SQL}
	\item[SQL] \textit{Structed Query Language}
	\item[ADMA] \textit{Association For Data Driven Marketing Advertising}
	\item[YOLO] \textit{You Only Look Once}
	\item[KMZ] \textit{Keyhole Markup Language Zip} 
	\item[XML]  \textit{Extensible Markup Language} 
	\item[HTML] \textit{Hypertext Markup Language} 
	\item[CSS] \textit{Cascading Style Sheets}
	\item[REST] \textit{Representational State Transfer}
	\item[HTTP] \textit{Hypertext Transfer Protocol}
	\item[JSON]  \textit{Javascript Object Notation}     
	\item[RDS] \textit{Relational Database Service}  
	\item[CPU] \textit{Central Process Unit} 
	\item[IaaS] \textit{Infrastructure as a Service}
	\item[SaaS] \textit{Software as a Service}
	\item[PaaS] \textit{Platform as a Service}
	\item[FaaS] \textit{Function as a Service}
	\item[RAM] \textit{Random Access Memory}
	\item[IBM] \textit{International Business Machines}
	\item[ARM] \textit{Advanced RISC Machine}
	\item[RISC]\textit{Reduced Instruction Set Computer}
	\item[GPU] \textit{Graphic Processor Unit}
	\item[SDK] \textit{Software Development Kit}
	\item[GCP] \textit{Google Cloud Platform}
	\item[TI]  Tecnologia da Informação
	\item[PyPi] \textit{Python Package Index}
	\item[MTV]	\textit{Model Template View}
	\item[MVC]   \textit{Model View Controller}
\end{siglas}

		\begin{simbolos}
  \item[$ \Omega $] Impedância
%  \item[$ V $] Tensão
%  \item[$ T $] Tesla
%  \item[$ B $] Campo Magnético
%  \item[$ bps $] bits por segundo
%  \item[$ Bps $] Bytes por segundo
%  \item[$ Hz $] Frequência 
%  \item[$ degrees/s $] Velocidade angular em graus por segundo
\end{simbolos}

		\pdfbookmark[0]{\contentsname}{toc}
		\tableofcontents*
			\cleardoublepage

		% Elementos textuais
		\textual
		\setlength{\parindent}{7ex}
		\setlength{\parskip}{1em}
		%\setlength{\baselineskip}{1.5cm}
		\chapter{Introdução}
\label{Cap:Intro}

% Capítulo 1: Introdução
\indent
\par O transporte público caracteriza-se como uma opção amplamente utilizada por pessoas a fim de garantirem suas necessidades de locomoção. Por possuírem um preço mais acessível e muitas das vezes serem mais rápidos e mais práticos, 65\% da população das capitais do Brasil utilizam essa forma de transporte, como aponta um estudo realizado pelo Instituto de Pesquisa Econômica Aplicada (Ipea).
\par Mesmo com esse cenário, em seu orçamento, a União só direcionou R\$ 707 milhões no ano de 2019, como indica a Lei Orçamentária Anual (LOA n° 13.808/2019), para a área de mobilidade urbana e trânsito, já especificamente para o transporte público coletivo, desse montante foram separados apenas R\$ 348 milhões, uma fatia de 0,01\% do orçamento total.
\par De acordo com uma estimativa do BNDES, em 2015, seria necessário investir mais de R\$ 234 bilhões em transporte público para resolver os problemas da área nas principais regiões metropolitanas do país, portanto, caso mantido o nível de investimento atual, levaria mais de 600 anos para atingir o montante proposto pelo BNDES.
\par Tendo em vista o baixo investimento frente a demanda, é de se esperar que o transporte público cause um nível elevado de insatisfação em seus usuários, o que o IPEA demonstrou em outra pesquisa realizada em 2011 e 2012, na qual o transporte público foi avaliado por mais de 60\% do público como "péssimo ou ruim".
\par Para entender a situação em que se encontra o transporte, o primeiro passo é olhar como o país se urbanizou. No último século, o Brasil passou por um intenso processo de industrialização, o que gerou um forte êxodo rural e principalmente uma migração da população do Nordeste para o Sul e Sudeste, onde se concentrou a produção industrial do país.
\par Esse crescimento populacional nas metrópoles foi acompanhado por uma grande valorização dos terrenos e moradias nas áreas centrais das cidades, e com isso vemos um efeito de gentrificação, afastando a classe trabalhadora para regiões mais periféricas, longe de onde se concentra a maior parte da oferta de emprego, o que gerou uma, ainda crescente, demanda por transporte. Tendo em vista esse cenário, a população começou a consumir cada vez mais carros populares, que contam com incentivo do governo para serem produzidos. Com isso, o que se vê nos ambientes urbanos são ruas congestionadas e ônibus lotados.
\par A situação não é diferente em Santo André, onde os ônibus destacam-se como o principal meio de transporte coletivo fornecido pelo município, contando com cerca de cinco milhões de passageiros por mês segundo dados da SA-Trans. Em seguida, o transporte ferroviário, mais especificamente as estações Celso Daniel, Prefeito Saladino e Utinga da linha 10 Turquesa da CPTM, ligando a cidade a São Paulo, garantem o segundo lugar no ranking, somando cerca de um milhão e oitocentos mil usuários em Fevereiro de 2020, segundo dados fornecidos pela CPTM, que administra a linha.

%\begin{comment}
\section{Justificativa}
\indent
\par Santo André é caracterizada como uma cidade de grande porte, possuindo 718.773 habitantes em 2019 segundo estimativa do IBGE (ver se precisa usar o dado do CENSO ou se pode usar essa estimativa - 676.407 hab). Entretanto, muitos cidadãos apresentam adversidades quando se trata da utilização dos ônibus da cidade, como o elevado tempo de espera e as altas taxas de ocupação interna, que lideram as reclamações do Reclame Aqui (procurar reclamações no PROCON). Os problemas citados ocorrem quando a oferta de carros não é ajustada adequadamente para atender as exigências da demanda de pessoas e das condições de trânsito.
\par Hoje, a cidade captura em tempo real e armazena vários dados relacionados ao transporte público. Desde a velocidade dos carros, ocorrências, horários de chegadas e partidas, informação de bilhetagem, entre outros. Porém, nem o município, nem a SA-Trans usam essas informações, o que possibilita que esse trabalho sirva de fundamento para uma informatização do transporte público local, e que permita melhorias no setor que não seriam possíveis apenas com as técnicas utilizadas atualmente.

%\end{comment}


\section{Objetivos}
\indent
\par Considerando o cenário apresentado, pretende-se criar um painel de controle para centralizar e agilizar a tomada de decisão, tendo como base dados coletados durante os trajetos de ônibus juntamente com outras informações extraídas da internet e imagens de câmera instaladas nos veículos. A ferramenta irá decidir de forma automatizada a disponibilidade de veículos em cada linha da cidade, controlando a demanda do serviço e auxiliando no dia a dia do transporte urbano na cidade.

\subsection{Objetivos Primários}

XXXXX
\\\\
XXXXX
\\\\
XXXXX

\subsection{Objetivos Secundários}

XXXXX
\\\\
XXXXX
\\\\
XXXXX

\section{Definição do Problema}
XXXXX
\\\\
XXXXX
\\\\
XXXXX
\\\\
XXXXX

\section{Questão Central da Pesquisa}

XXXXX

\section{Contribuições do Trabalho}

XXXXX

\section{Panorama Econômico}
XXXXX
\\\\
XXXXX

\subsection{Mercados}

XXXXX
\\\\
XXXXX
\\\\
XXXXX

\subsection{Oportunidades}

XXXXX
\\\\
XXXXX
\\\\
XXXXX


\section{Sustentabilidade e Impacto Social}

XXXXX




\section{Impactos sociais}

XXXXX
 \cleardoublepage  % quando usa twoside
		\chapter{Revisão bibliográfica}
\label{Cap:RevisaoBibliografica}
\newcommand{\WidthAlgumaCoisa}{6.5 cm}

% Capítulo 2: Revisão Bibliográfica
XXXXXX.

\section{Assunto 1}

XXXXXX
\\\\
XXXXXX

\subsection{SubAssunto 1}

XXXXXX
\\\\
XXXXXX

\section{Assunto 2}

XXXXXX
\\\\
XXXXXX

\subsection{SubAssunto 2}

XXXXXX
\\\\
XXXXXX

\section{Assunto 3}

XXXXXX
\\\\
XXXXXX

\subsection{SubAssunto 3}

XXXXXX
\\\\
XXXXXX

\section{Assunto 4}

XXXXXX
\\\\
XXXXXX

\subsection{SubAssunto 4}

XXXXXX
\\\\
XXXXXX


 \cleardoublepage
		\chapter{Metodologia}
\label{Cap:MateriaisMetodos}
% Capítulo 3: Materiais e Métodos (ou Metodologia)

\indent
\par A fim de solucionar o problema identificado, foram definidos métodos e estratégias com o objetivo de criar um fluxo de trabalho. A figura \ref{DiagramaDeBlocosIcones} fornece um diagrama ilustrativo e simplificado da solução proposta.

\begin{figure}[H]
    \centering
    \caption{Ilustração da solução}
    \includegraphics[width=1.0\linewidth]{Imagens/DiagramaDeBlocosIcones.png}
    \caption*{Fonte: Arquivo dos autores (2020)}
    \label{DiagramaDeBlocosIcones}
\end{figure}

\indent
\par Primeiramente alguns dados dos ônibus como por exemplo velocidade, tempo de parada, linhas, posição dos veículos seriam coletados juntamente com imagens do interior dos veículos, que passariam por um algoritmo de reconhecimento para identificar a lotação dos mesmos. Paralelamente, informações obtidas por meio de APIs externas enriqueceriam o conjunto de dados, que por sua vez passaria por um tratamento e um treinamento. Após esse processo algumas informações serão direcionadas diretamente para o painel de controle enquanto outras passarão pela inteligencia artificial para depois serem colocadas no \textit{dashboard}.

\section{Aquisição de dados externos por meio de tarefas}
\indent
\par No que diz respeito à aquisição de dados para alimentação do \textit{dashboard} e sucessiva análise, foi utilizada como principal fonte, a API pública da SP-Trans, mais especificamente os \textit{endpoints} “Parada”, “Posicao” e “KMZ”. Cada \textit{endpoint} é responsável por trazer informações como as paradas dos corredores dos ônibus de São Paulo, a posição dos veículos, assim como sua respectiva linha e frota em um dado instante e um arquivo KMZ contendo informações de velocidade das linhas, respectivamente.
\indent
\par Vale ressaltar que para a aquisição dos dados do arquivo KMZ, foi realizada uma quebra do arquivo, que originalmente possui as informações em formato XML. Esse procedimento foi necessário para obter todas as informações já tratadas do arquivo em formato JSON para carga do banco de dados.
\indent
\par Adicionalmente a API da SP-Trans, são coletados dados de outras fontes, como Climatempo, Direto dos Trens e \textit{Tickets for Fun}, que são responsáveis por dados como condições meteorológicas, situação das linhas de trens e metrôs em tempo real e localização e data de eventos, respectivamente.
\indent
\par Tendo em vista que todos os dados são coletados com informações em tempo real, é necessário que eles sejam armazenados para que se construa um histórico a fim de se analisar os dados futuramente. Com esse objetivo, foi utilizado o Celery, um módulo Python que permite a criação de tarefas assíncronas e agendadas. Tais tarefas foram modeladas para coletar as informações das fontes mencionadas anteriormente em um determinado período de tempo de forma assíncrona e salvá-las em um banco de dados Postgres, criado exclusivamente para o projeto.
\indent
\par Paralelamente à utilização do Celery, foi utilizado um serviço da AWS, o \textit{Amazon Simple Queue Service} (SQS), que recebe as tarefas agendadas do Celery e as administra em uma fila, enviando novamente para execução quando for a hora certa. Dessa maneira, as informações em tempo real passaram a ser extraídas ciclicamente por meio das tarefas e salvas em um banco de dados dedicado, gerando um histórico para análise.

\section{Criação de gráficos com Django}
\indent
\par Para a criação do dashboard do projeto que exibe informações extraídas a partir dos dados disponíveis no banco de dados, foi utilizado o Django, um \textit{framework} Python de desenvolvimento rápido para \textit{web}.
\indent
\par O Django utiliza o padrão \textit{model-template-view}, que fornece todas as ferramentas necessárias para o desenvolvimento \textit{web}, desde a criação de um modelo associado ao banco de dados, até o processamento de requisições e criação de páginas \textit{web} dinâmicas de forma robusta e simples. Um de seus principais diferenciais é a padronização dos aplicativos e portabilidade dos mesmos para outros projetos. Isso possibilita uma fácil implementação de módulos e aplicativos externos.
\indent
\par Assim sendo, um dos principais aplicativos utilizados nesse projeto foi o \textit{django-plotly-dash}, que possibilita uma fácil criação de gráficos e painéis que são atualizados em tempo real utilizando apenas código Python, sem a necessidade de escrever código HTML, CSS ou Javascript. Isso possibilitou uma integração direta dos modelos criados no Django com os gráficos. A figura \ref{printDashboard} mostra alguns exemplos de gráficos desenvolvidos com o \textit{django-plotly-dash} para o projeto.

\begin{figure}[H]
    \centering
    \caption{Gráficos criados pelo django-plotly-dash}
    \includegraphics[width=1.0\linewidth]{Imagens/printDashboard.jpeg}
    \caption*{Fonte: Arquivo dos autores (2020)}
    \label{printDashboard}
\end{figure}
\indent
\par Quanto aos modelos mencionados anteriormente, foi criado um modelo para cada informação extraída das APIs externas, estando eles associados à posição dos ônibus, velocidade das vias, linhas e paradas de ônibus, informações das linhas de trens e metrôs, informações meteorológicas e eventos. Todos os modelos estão associados a um banco de dados Postgres, que por possuir uma melhores integração com Python e Django, além de se tratar de um banco de dados relacional e rápido, foi a opção escolhida para o projeto.

\section{APIs com Django REST \textit{Framework}}
\indent
\par O Django REST \textit{Framework} é uma biblioteca para o \textit{Framework} Django que disponibiliza funcionalidades para implementar APIs REST de forma extremamente rápida e fácil. É extremamente simples configurar e criar rotas que aceitam todos os verbos HTTP se comunicando diretamente com o banco de dados e atendendo ao CRUD (\textit{Create} (Criação), \textit{Read} (Consulta), \textit{Update} (Atualização) e \textit{Delete} (Destruição)).
\indent
\par Outra grande vantagem, além da facilidade de implementação, é a criação automática de uma plataforma \textit{web} que centraliza todas as possibilidades de requisições de todas as rotas disponíveis. Nela é possível escolher a forma que deseja visualizar os dados, seja como JSON ou XML o tipo de retorno.
\indent
\par Com as API’s implementadas, é possível acessar os dados da aplicação de forma independente do \textit{dashboard}. Assim, outras pessoas podem ter acesso a esses dados para novas e futuras análises. Alguns dos \textit{endpoints} criados disponibilizam dados como de velocidade, posição, linhas, paradas, metrô, trem e clima.

\begin{figure}[H]
    \centering
    \caption{Rotas das APIs dentro da plataforma \textit{web} disponibilizada pelo Django}
    \includegraphics[width=1.0\linewidth]{Imagens/rotasDisponiveis.png}
    \caption*{Fonte: Arquivo dos autores (2020)}
    \label{rotasDisponiveis}
\end{figure}

\begin{figure}[H]
    \centering
    \caption{Exemplo de retorno JSON do \textit{endpoint} /api/onibus-lotacao}
    \includegraphics[width=1.0\linewidth]{Imagens/onibusLotacaoExemplo.png}
    \caption*{Fonte: Arquivo dos autores (2020)}
    \label{onibusLotacaoExemplo}
\end{figure}
\indent
\par Na figura \ref{rotasDisponiveis}, podemos visualizar uma lista com todas as rotas disponíveis. Um exemplo de requisição nessas rotas pode ser visualizado na imagem \ref{onibusLotacaoExemplo}, onde é realizada uma requisição do tipo \textit{GET} no \textit{endpoint} ‘onibus-lotacao’. 
\indent
\par Podemos ver que a resposta foi do tipo JSON, retornando todas as informações disponíveis como, id, id\_onibus, id\_linha, lotação, latitude, longitude, data\_inclusao.

\section{\textit{Endpoints}}
\indent
\par Na imagem \ref{rotasDisponiveis}, podemos visualizar todos os \textit{endpoints} disponíveis, cada um deles possui suas informações e funções. Abaixo, uma breve descrição de cada uma delas, e como utilizá-las com os verbos \textit{GET} e \textit{POST}, assim como uma prévia do retorno das informações.
\indent
\par Foi escolhido apresentar as requisições \textit{GET} e \textit{POST} pois são através delas que o dashboard consome os dados e que as APIs externas salvam as informações dentro do nosso banco de dados.

\subsection{/api/onibus-lotacao/}
\textit{GET}
\begin{lstlisting}
{
    "id": 1,
    "id_onibus": 96587,
    "id_linha": 2215,
    "lotacao": "cheio",
    "latitude": "45.658000000000001",
    "longitude": "12.987000000000000",
    "data_inclusao": "2020-10-09T19:41:53.075905Z"
}    
\end{lstlisting}
\textit{POST}
\begin{lstlisting}
{
    "img": caminho_img,
    "id_onibus": id_onibus,
    "id_linha": id_linha,
    "latitude": latitude,
    "longitude": longitude            
}
\end{lstlisting}
\subsection{/api/onibus-posicao/}
\textit{GET}
\begin{lstlisting}
{
    "quantidade": 11992
}
\end{lstlisting}
\textit{POST}
\begin{lstlisting}
{
    "o": [
        {
            "id_onibus": id_onibus,
            "onibus_deficiente": onibus_deficiente,
            "horario_atualizacao_localizacao": horario_atualizacao_localizacao,
            "latitude": latitude,
            "longitude": longitude,
            "id_linha": id_linha,
            "frota": frota,
        }
    ]
}
\end{lstlisting}
\subsection{/api/onibus-velocidade/}
\textit{GET}
\begin{lstlisting}
{
    "id": 1067508,
    "nome": "BUTANTA (BAIRRO - CENTRO)",
    "vel_trecho": 27,
    "vel_via": 27,
    "trecho": "de R. AMARO CAVALHEIRO ate R. PAES LEME",
    "extensao": 650,
    "tempo": "00:01",
    "coordenadas": [
        {
            "latitude": "-23.567653",
            "longitude": "-46.695722",
            "id": 6217
        },
        {
            "latitude": "-23.568001",
            "longitude": "-46.696452",
            "id": 6218
        },
        {
            "latitude": "-23.568063",
            "longitude": "-46.696595",
            "id": 6219
        }
}
\end{lstlisting}
\textit{POST}
\begin{lstlisting}
{
    "o": [
        {
            "name": nome,
            "description": {
                'vel_trecho': vel_trecho,
                'vel_via': vel_via,
                'trecho': trecho,
                'extensao': extensao,
                'tempo': tempo
            },
            "coordinates": {
                'lat': latitude,
                'lon': longitude
            }
        }
    ]
}
\end{lstlisting}
\subsection{/api/linhas/}
\textit{GET}
\begin{lstlisting}
{
    "id_linha": 264,
    "letreiro": "509J-10",
    "sentido": 1,
    "letreiro_destino": "PQ. IBIRAPUERA",
    "letreiro_origem": "JD. SELMA"
}
\end{lstlisting}
\textit{POST}
\begin{lstlisting}
{
    "l": [
        {
            "id_linha": id_linha,
            "letreiro": letreiro,
            "sentido": sentido,
            "letreiro_destino": letreiro_destino,
            "letreiro_origem": letreiro_origem,
        }
    ]
}    
\end{lstlisting}
\subsection{/api/paradas/}
\textit{GET}
\begin{lstlisting}
{
    "id": 1,
    "id_parada": 4203724,
    "nome": "",
    "endereco": "R. Agamenon Pereira da Silva",
    "latitude": "-23.692865000000001",
    "longitude": "-46.778350000000003"
}
\end{lstlisting}
\textit{POST}
\begin{lstlisting}
{
    "p": [
        {
            "id_parada": id_parada,
            "nome": nome,
            "endereco": endereco,
            "latitude": latitude,
            "longitude": longitude,
        }
    ]
}
\end{lstlisting}
\subsection{/api/trens/}
\textit{GET}
\begin{lstlisting}
{
    "id": 1,
    "id_linha": 1,
    "data_ocorrencia": "2020-10-07T07:42:01.039607Z",
    "descricao": null,
    "ultima_atualizacao": "2020-10-07T21:14:01.162598Z",
    "situacao": "Operacao Normal"
}
\end{lstlisting}
\textit{POST}
\begin{lstlisting}
{
    "t": [
        {
            "id_linha": id_linha,
            "data_ocorrencia": data_ocorrencia,
            "descricao": descricao,
            "ultima_atualizacao": ultima_atualizacao,
            "situacao": situacao,
        }
    ]
}
\end{lstlisting}
\subsection{/api/climatempo/}
\textit{GET}
\begin{lstlisting}
{
    "id_cidade": 3477,
    "temperatura": "27.00",
    "direcao_vento": "S",
    "velocidade_vento": "9.00",
    "umidade": "66.00",
    "condicao": "Nuvens esparsas",
    "pressao": "1015.00",
    "sensacao": "28.00",
    "date": "2020-10-07T21:14:34.451818Z"
}
\end{lstlisting}
\textit{POST}
\begin{lstlisting}
{
    "ct": [
        {
            "id_cidade": id_cidade,
            "temperatura": temperatura,
            "direcao_vento": direcao_vento,
            "velocidade_vento": velocidade_vento,
            "umidade": umidade,
            "condicao": condicao,
            "pressao": pressao,
            "sensacao": sensacao,
        }
    ]
}
\end{lstlisting}
\subsection{/api/eventos/}
\textit{GET}
\begin{lstlisting}
{
    "id": 26,
    "nome": "Maria Bethania",
    "link": "http://premier.ticketsforfun.com.br/shows/show.aspx?sh=MARIBUB19",
    "data_info": "Sao Paulo no UnimedHall 27 de junho",
    "data": "2020-06-27",
    "endereco": "Av. das Nacoes Unidas, 17955 - Vila Almeida, Sao Paulo - SP, 04795-100, Brazil",
    "latitude": "-23.647672600000000",
    "longitude": "-46.723812100000004",
    "data_inclusao": "2020-10-14T00:17:58.612131Z"
}
\end{lstlisting}
\textit{POST}
\begin{lstlisting}
{
    "e": [
        {
            "nome": nome,
            "link": link,
            "data_info": data_info,
            "data": data,
            "endereco": endereco,
            "latitude": latitude,
            "longitude": longitude
        }
    ]
}
\end{lstlisting}

\section{Infraestrutura de nuvem}
\indent
\par Pela necessidade de estar disponível a qualquer momento e de capturar o máximo de informação possível, decidimos disponibilizar o nosso sistema na nuvem, utilizando a plataforma da Amazon, principalmente os recursos da \textit{Elastic Cloud Compute} (EC2). A nossa infraestrutura gira em torno de uma máquina Linux, da categoria t2.medium, que se provou o mínimo necessário para executar o nosso sistema e aguentar a carga de algumas visitas simultâneas.
\indent
\par Começando pela máquina, as instâncias EC2 t2.medium são máquinas com 2 núcleos dedicados, 4GB de memória RAM, indicadas tanto para o uso como plataforma de desenvolvimento quanto para servidores \textit{web}. Entre suas características temos processadores com um \textit{clock} relativamente alto e ao mesmo tempo um custo baixo em comparação com as outras soluções oferecidas pela Amazon, como instâncias t3 ou A1. 
\indent
\par Decidimos usar a instância t2.medium após realizar testes tanto com a t2.micro, que é gratuita para as 750 primeiras horas por mês, quanto a t2.small, que é totalmente paga, mas mais econômica que a t2.medium. A modalidade micro apresentou dificuldades logo nos primeiros instantes, apesar de ser capaz de executar o dashboard em Django paralelo ao banco de dados PostgreSQL, quando começamos a rodar as tasks do Celery para captura de dados, o processamento disponível para ela não foi o suficiente, atingindo 100\% de uso do núcleo disponível e deixando o sistema inutilizável, mesmo com uma única pessoa acessando. Já a instância t2.small se mostrou capaz de executar as \textit{tasks} simultaneamente ao \textit{dashboard} e o banco de dados, porém ao começar a receber acessos no \textit{dashboard} o sistema não conseguia dar conta das requisições e apresentava uma constante lentidão. 
\indent
\par Com isso, chegamos a conclusão que a instância mínima para rodarmos o nosso projeto com segurança é a t3.medium, e mesmo assim rapidamente consumimos todo o seu armazenamento em pouco menos de uma semana, após isso expandimos ele de 7GB para 50GB, o que nos daria uma margem de segurança até a conclusão do projeto.



 \cleardoublepage
		\chapter{Resultados Obtidos}
\label{Cap:Resultados}
\newcommand{\EscalaAlgumaCoisa}{0.6}

% Capítulo 4: Resultados

xxxxx
\\\\
xxxxx 



%\begin{figure}[H]
%    \centering
%    \caption{Resultados da ferramenta AutoAI - IBM}
%    \includegraphics[width=1.0\linewidth]{Imagens/autoai-results.png}
%    \caption*{Fonte: Arquivo dos autores (2019)}
%    \label{autoai-results}
%\end{figure}



\section{Equações}

xxxxx

\section{Códigos Fonte de Programação}

xxxxx

 \cleardoublepage
		\chapter{Conclusões}
\label{Cap:Conclusoes}

% Capítulo 5: Conclusões

xxxxx
\\\\
xxxxx




 \cleardoublepage
	
		

		% Apêndices
		%\begin{apendicesenv}	
        	%\partapendices
			%\chapter{APÊNDICE}
 \label{app:apendiceA}
\section{\textit{ENDPOINTS}}
\subsection{/api/onibus-lotacao/}
\textit{GET}
\begin{lstlisting}
{
    "id": 1,
    "id_onibus": 96587,
    "id_linha": 2215,
    "lotacao": "cheio",
    "latitude": "45.658000000000001",
    "longitude": "12.987000000000000",
    "data_inclusao": "2020-10-09T19:41:53.075905Z"
}    
\end{lstlisting}
\textit{POST}
\begin{lstlisting}
{
    "img": caminho_img,
    "id_onibus": id_onibus,
    "id_linha": id_linha,
    "latitude": latitude,
    "longitude": longitude            
}
\end{lstlisting}
\subsection{/api/onibus-posicao/}
\textit{GET}
\begin{lstlisting}
{
    "quantidade": 11992
}
\end{lstlisting}
\textit{POST}
\begin{lstlisting}
{
    "o": [
        {
            "id_onibus": id_onibus,
            "onibus_deficiente": onibus_deficiente,
            "horario_atualizacao_localizacao": horario_atualizacao_localizacao,
            "latitude": latitude,
            "longitude": longitude,
            "id_linha": id_linha,
            "frota": frota,
        }
    ]
}
\end{lstlisting}
\subsection{/api/onibus-velocidade/}
\textit{GET}
\begin{lstlisting}
{
    "id": 1067508,
    "nome": "BUTANTA (BAIRRO - CENTRO)",
    "vel_trecho": 27,
    "vel_via": 27,
    "trecho": "de R. AMARO CAVALHEIRO ate R. PAES LEME",
    "extensao": 650,
    "tempo": "00:01",
    "coordenadas": [
        {
            "latitude": "-23.567653",
            "longitude": "-46.695722",
            "id": 6217
        },
        {
            "latitude": "-23.568001",
            "longitude": "-46.696452",
            "id": 6218
        },
        {
            "latitude": "-23.568063",
            "longitude": "-46.696595",
            "id": 6219
        }
}
\end{lstlisting}
\textit{POST}
\begin{lstlisting}
{
    "o": [
        {
            "name": nome,
            "description": {
                'vel_trecho': vel_trecho,
                'vel_via': vel_via,
                'trecho': trecho,
                'extensao': extensao,
                'tempo': tempo
            },
            "coordinates": {
                'lat': latitude,
                'lon': longitude
            }
        }
    ]
}
\end{lstlisting}
\subsection{/api/linhas/}
\textit{GET}
\begin{lstlisting}
{
    "id_linha": 264,
    "letreiro": "509J-10",
    "sentido": 1,
    "letreiro_destino": "PQ. IBIRAPUERA",
    "letreiro_origem": "JD. SELMA"
}
\end{lstlisting}
\textit{POST}
\begin{lstlisting}
{
    "l": [
        {
            "id_linha": id_linha,
            "letreiro": letreiro,
            "sentido": sentido,
            "letreiro_destino": letreiro_destino,
            "letreiro_origem": letreiro_origem,
        }
    ]
}    
\end{lstlisting}
\subsection{/api/paradas/}
\textit{GET}
\begin{lstlisting}
{
    "id": 1,
    "id_parada": 4203724,
    "nome": "",
    "endereco": "R. Agamenon Pereira da Silva",
    "latitude": "-23.692865000000001",
    "longitude": "-46.778350000000003"
}
\end{lstlisting}
\textit{POST}
\begin{lstlisting}
{
    "p": [
        {
            "id_parada": id_parada,
            "nome": nome,
            "endereco": endereco,
            "latitude": latitude,
            "longitude": longitude,
        }
    ]
}
\end{lstlisting}
\subsection{/api/trens/}
\textit{GET}
\begin{lstlisting}
{
    "id": 1,
    "id_linha": 1,
    "data_ocorrencia": "2020-10-07T07:42:01.039607Z",
    "descricao": null,
    "ultima_atualizacao": "2020-10-07T21:14:01.162598Z",
    "situacao": "Operacao Normal"
}
\end{lstlisting}
\textit{POST}
\begin{lstlisting}
{
    "t": [
        {
            "id_linha": id_linha,
            "data_ocorrencia": data_ocorrencia,
            "descricao": descricao,
            "ultima_atualizacao": ultima_atualizacao,
            "situacao": situacao,
        }
    ]
}
\end{lstlisting}
\subsection{/api/climatempo/}
\textit{GET}
\begin{lstlisting}
{
    "id_cidade": 3477,
    "temperatura": "27.00",
    "direcao_vento": "S",
    "velocidade_vento": "9.00",
    "umidade": "66.00",
    "condicao": "Nuvens esparsas",
    "pressao": "1015.00",
    "sensacao": "28.00",
    "date": "2020-10-07T21:14:34.451818Z"
}
\end{lstlisting}
\textit{POST}
\begin{lstlisting}
{
    "ct": [
        {
            "id_cidade": id_cidade,
            "temperatura": temperatura,
            "direcao_vento": direcao_vento,
            "velocidade_vento": velocidade_vento,
            "umidade": umidade,
            "condicao": condicao,
            "pressao": pressao,
            "sensacao": sensacao,
        }
    ]
}
\end{lstlisting}
\subsection{/api/eventos/}
\textit{GET}
\begin{lstlisting}
{
    "id": 26,
    "nome": "Maria Bethania",
    "link": "http://premier.ticketsforfun.com.br/shows/show.aspx?sh=MARIBUB19",
    "data_info": "Sao Paulo no UnimedHall 27 de junho",
    "data": "2020-06-27",
    "endereco": "Av. das Nacoes Unidas, 17955 - Vila Almeida, Sao Paulo - SP, 04795-100, Brazil",
    "latitude": "-23.647672600000000",
    "longitude": "-46.723812100000004",
    "data_inclusao": "2020-10-14T00:17:58.612131Z"
}
\end{lstlisting}
\textit{POST}
\begin{lstlisting}
{
    "e": [
        {
            "nome": nome,
            "link": link,
            "data_info": data_info,
            "data": data,
            "endereco": endereco,
            "latitude": latitude,
            "longitude": longitude
        }
    ]
}
\end{lstlisting}
 

			
%			\include{apendicec}
		%\end{apendicesenv}
		
		% Anexos
		%\begin{anexosenv}
			%\partanexos % Imprime uma página indicando o início dos anexos
		%\end{anexosenv}
\nocite{}
\bibliography{referencias}	
\end{document}