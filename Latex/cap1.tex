\chapter{Introdução}
\label{Cap:Intro}

% Capítulo 1: Introdução

\section{Transporte público}
\indent
\par O transporte público caracteriza-se como uma opção amplamente utilizada por pessoas a fim de garantirem suas necessidades de locomoção. Por possuir um preço mais acessível e muitas vezes ser mais rápido e prático, 65\% da população das capitais do Brasil utiliza essa forma de transporte, como aponta um estudo realizado pelo Instituto de Pesquisa Econômica Aplicada (Ipea) \cite{Peduzzi2011}.

\par Foi direcionado o valor de R\$ 707 milhões no ano de 2019 pela União para a área de mobilidade urbana e trânsito, como indica a Lei Orçamentária Anual (LOA n° 13.808/2019). Desse montante, especificamente para o transporte público coletivo, foram separados apenas R\$ 348 milhões, uma fatia de 0,01\% do orçamento total da União \cite{NTUrbano}.
\par De acordo com uma estimativa do BNDES, em 2015, seria necessário investir mais de R\$ 234 bilhões em transporte público para resolver os problemas da área nas principais regiões metropolitanas do país, portanto, caso mantido o nível de investimento atual, levaria mais de 600 anos para atingir o montante proposto pelo BNDES \cite{Santos2015}.

\par Tendo em vista o baixo investimento frente a demanda, é esperado que o transporte público gere um elevado nível de insatisfação em seus usuários, o que o IPEA demonstrou em outra pesquisa realizada em 2011 e 2012, na qual o transporte público foi avaliado por mais de 60\% do público como "péssimo ou ruim"  \cite{Santos2015}.
\par Para entender a situação em que se encontra o transporte, o primeiro passo é olhar como o país se urbanizou. No último século, o Brasil passou por um intenso processo de industrialização, o que gerou um forte êxodo rural e principalmente uma migração da população do Nordeste para o Sul e Sudeste, onde se concentrou a produção industrial do país.
\par Esse crescimento populacional nas metrópoles foi acompanhado por uma grande valorização dos terrenos e moradias nas áreas centrais das cidades. Com isso é observado um efeito de gentrificação, afastando a classe trabalhadora para regiões mais periféricas, longe de onde se concentra a maior parte da oferta de emprego, o que gerou uma demanda por transporte, crescente até os dias atuais.
Tendo em vista esse cenário, a população começou a consumir cada vez mais carros populares, que contam com incentivo do governo para serem produzidos. Com isso, o que se vê nos ambientes urbanos são ruas congestionadas e ônibus lotados \cite{PenaSD}.


\section{Justificativa}
\indent
\par O presente trabalho tem por motivação a grande quantidade de cidadãos de São Paulo que apresentam adversidades quando se trata da utilização dos ônibus da cidade, como o elevado tempo de espera e as altas taxas de ocupação interna, que lideram as reclamações do Reclame Aqui \cite{ReclameAqui1} \cite{ReclameAqui2}. Além disso, a criação de um processo inteligente e automatizado que auxilie os gestores da SPTrans (empresa responsável pelo gerenciamento dos transportes do município de São Paulo) no ajuste das frotas de ônibus justificam esse projeto.

\section{Mercados}
\indent
\par Ainda olhando para o transporte público de São Paulo, hoje o sistema todo, composto de ônibus, trens, metrô e outros modais, transporta mais de 17 milhões de passageiros diariamente na capital \cite{G1SaoPaulo}, sendo que dessa quantidade, 7 milhões utilizam as linhas de ônibus da cidade, que hoje conta com mais de 1.300 linhas \cite{MobilidadeSampa}, sendo as principais a Terminal Bandeira / Terminal Varginha, Terminal Jardim  ngela / Metrô Santa Cruz e Hospital Itaim / Guaianazes , que juntas transportam cerca de 100 mil passageiros por dia \cite{Viatrolebus}.
\par Hoje, a SPTrans disponibiliza uma API, atualizada em tempo real, com informações de localização dos veículos, velocidades, pontos de parada e quantidade de ônibus ativos. Apesar de provavelmente a empresa contar com uma solução proprietária, decidimos coletar essas informações e enriquece-las com outros dados disponíveis para desenvolver o nosso trabalho.

\section{Objetivos}
\indent
\par Tendo em vista o cenário apresentado, pretende-se criar um \textit{dashboard} para centralizar informações e agilizar a tomada de decisão, tendo como base dados coletados durante os trajetos de ônibus, informações extraídas de APIs públicas e imagens de câmeras instaladas nos veículos. A ferramenta irá apoiar os gestores da SPTrans no controle do fluxo de tráfego, analisando a disponibilidade de veículos e sugerindo uma inclusão ou remoção de carros em cada linha monitorada.
\par Além disso, o \textit{dashboard} será atualizado em tempo real, exibindo gráficos e indicadores de maneira sucinta, com a possibilidade de aplicação de filtros nas consultas. Considerando o volume de passageiros para um determinado dia e horário somados a dados externos, a ferramenta irá utilizar algoritmos de inteligência artificial para identificar padrões históricos e prever futuras demandas nas frotas de ônibus, facilitando o controle do serviço e auxiliando no dia a dia do transporte urbano na cidade.


\section{Contribuições do trabalho}
\indent
\par Este trabalho tem como contribuição a melhoria na tomada de decisões da administradora de ônibus (SPTrans) onde será possível, de forma centralizada, a visualização de dados e gráficos por meio de um dashboard. Além disso, pode-se citar como contribuição a diminuição do tempo de espera e ocupação dos transportes, resultando em uma melhor qualidade de vida para a população de São Paulo.

\section{Questão central da pesquisa e seus impactos}
\indent
\par Os benefícios esperados com esse trabalho de conclusão de curso são proporcionar uma melhor qualidade locomotiva dos ônibus de São Paulo e proporcionar uma melhor gestão aos envolvidos da SPTrans, otimizando e inovando a forma de ver os dados.

\indent
\par O transporte público, mais especificamente, os ônibus, podem causar estresse. A sensação pode estar relacionada a fatores como as condições do transporte, o desconforto de estar entre muitas pessoas e o elevado tempo de espera.
\par Com os dados disponíveis e as aplicações de IA (Inteligência Artificial) implementadas nas câmeras dos veículos de cada linha, os gestores e responsáveis poderão ver a situação de lotação e do tempo de espera através do dashboard. Numa situação crítica, eles poderão solicitar um aumento da frota e reduzir o desconforto de andar em um veículo com excesso de pessoas além de reduzir o tempo de espera. Essas ações impactam diretamente na saúde das pessoas reduzindo o estresse gerado pelos transportes.
