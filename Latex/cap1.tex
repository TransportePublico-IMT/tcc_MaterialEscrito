\chapter{Introdução}
\label{Cap:Intro}

% Capítulo 1: Introdução
\indent
\par O transporte público caracteriza-se como uma opção amplamente utilizada por pessoas a fim de garantirem suas necessidades de locomoção. Por possuírem um preço mais acessível e muitas das vezes serem mais rápidos e mais práticos, 65\% da população das capitais do Brasil utilizam essa forma de transporte, como aponta um estudo realizado pelo Instituto de Pesquisa Econômica Aplicada (Ipea).
\par Mesmo com esse cenário, em seu orçamento, a União só direcionou R\$ 707 milhões no ano de 2019, como indica a Lei Orçamentária Anual (LOA n° 13.808/2019), para a área de mobilidade urbana e trânsito, já especificamente para o transporte público coletivo, desse montante foram separados apenas R\$ 348 milhões, uma fatia de 0,01\% do orçamento total.
\par De acordo com uma estimativa do BNDES, em 2015, seria necessário investir mais de R\$ 234 bilhões em transporte público para resolver os problemas da área nas principais regiões metropolitanas do país, portanto, caso mantido o nível de investimento atual, levaria mais de 600 anos para atingir o montante proposto pelo BNDES.
\par Tendo em vista o baixo investimento frente a demanda, é de se esperar que o transporte público cause um nível elevado de insatisfação em seus usuários, o que o IPEA demonstrou em outra pesquisa realizada em 2011 e 2012, na qual o transporte público foi avaliado por mais de 60\% do público como "péssimo ou ruim".
\par Para entender a situação em que se encontra o transporte, o primeiro passo é olhar como o país se urbanizou. No último século, o Brasil passou por um intenso processo de industrialização, o que gerou um forte êxodo rural e principalmente uma migração da população do Nordeste para o Sul e Sudeste, onde se concentrou a produção industrial do país.
\par Esse crescimento populacional nas metrópoles foi acompanhado por uma grande valorização dos terrenos e moradias nas áreas centrais das cidades, e com isso vemos um efeito de gentrificação, afastando a classe trabalhadora para regiões mais periféricas, longe de onde se concentra a maior parte da oferta de emprego, o que gerou uma, ainda crescente, demanda por transporte. Tendo em vista esse cenário, a população começou a consumir cada vez mais carros populares, que contam com incentivo do governo para serem produzidos. Com isso, o que se vê nos ambientes urbanos são ruas congestionadas e ônibus lotados.
\par A situação não é diferente em Santo André, onde os ônibus destacam-se como o principal meio de transporte coletivo fornecido pelo município, contando com cerca de cinco milhões de passageiros por mês segundo dados da SA-Trans. Em seguida, o transporte ferroviário, mais especificamente as estações Celso Daniel, Prefeito Saladino e Utinga da linha 10 Turquesa da CPTM, ligando a cidade a São Paulo, garantem o segundo lugar no ranking, somando cerca de um milhão e oitocentos mil usuários em Fevereiro de 2020, segundo dados fornecidos pela CPTM, que administra a linha.

%\begin{comment}
\section{Justificativa}
\indent
\par Santo André é caracterizada como uma cidade de grande porte, possuindo 718.773 habitantes em 2019 segundo estimativa do IBGE (ver se precisa usar o dado do CENSO ou se pode usar essa estimativa - 676.407 hab). Entretanto, muitos cidadãos apresentam adversidades quando se trata da utilização dos ônibus da cidade, como o elevado tempo de espera e as altas taxas de ocupação interna, que lideram as reclamações do Reclame Aqui (procurar reclamações no PROCON). Os problemas citados ocorrem quando a oferta de carros não é ajustada adequadamente para atender as exigências da demanda de pessoas e das condições de trânsito.
\par Hoje, a cidade captura em tempo real e armazena vários dados relacionados ao transporte público. Desde a velocidade dos carros, ocorrências, horários de chegadas e partidas, informação de bilhetagem, entre outros. Porém, nem o município, nem a SA-Trans usam essas informações, o que possibilita que esse trabalho sirva de fundamento para uma informatização do transporte público local, e que permita melhorias no setor que não seriam possíveis apenas com as técnicas utilizadas atualmente.

%\end{comment}


\section{Objetivos}
\indent
\par Considerando o cenário apresentado, pretende-se criar um painel de controle para centralizar e agilizar a tomada de decisão, tendo como base dados coletados durante os trajetos de ônibus juntamente com outras informações extraídas da internet e imagens de câmera instaladas nos veículos. A ferramenta irá decidir de forma automatizada a disponibilidade de veículos em cada linha da cidade, controlando a demanda do serviço e auxiliando no dia a dia do transporte urbano na cidade.

\subsection{Objetivos Primários}

XXXXX
\\\\
XXXXX
\\\\
XXXXX

\subsection{Objetivos Secundários}

XXXXX
\\\\
XXXXX
\\\\
XXXXX

\section{Definição do Problema}
XXXXX
\\\\
XXXXX
\\\\
XXXXX
\\\\
XXXXX

\section{Questão Central da Pesquisa}

XXXXX

\section{Contribuições do Trabalho}

XXXXX

\section{Panorama Econômico}
XXXXX
\\\\
XXXXX

\subsection{Mercados}

XXXXX
\\\\
XXXXX
\\\\
XXXXX

\subsection{Oportunidades}

XXXXX
\\\\
XXXXX
\\\\
XXXXX


\section{Sustentabilidade e Impacto Ambiental}

XXXXX




\section{Impactos Sociais}

XXXXX
